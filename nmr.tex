\documentclass[12pt]{article}

\usepackage{fancyhdr}
\usepackage{geometry}
\usepackage{ucs}
\usepackage[utf8x]{inputenc}
\usepackage[T1]{fontenc}
\usepackage[german]{babel}
\usepackage{amsmath,amssymb,amstext}
\usepackage{hyperref}
\usepackage{cancel}
\usepackage{dsfont}
\usepackage{physics}
\usepackage{lmodern}
\usepackage{enumerate}
\usepackage{enumitem}
\usepackage{graphicx}
\usepackage{listings, color}
\usepackage[labelfont=bf]{caption}
\usepackage{titling}

\lstset{basicstyle=\scriptsize} %Quellcode mit Umlauten und ganz klein
\lstset{literate=
  {Ö}{{\"O}}1
  {Ä}{{\"A}}1
  {Ü}{{\"U}}1
  {ß}{{\ss}}2
  {ü}{{\"u}}1
  {ä}{{\"a}}1
  {ö}{{\"o}}1
}


%Geometrie----------------------------------------------------------------------------------------------------------

\geometry{a4paper, top=25mm, left=15mm, right=15mm, bottom=25mm,headsep=10mm, footskip=10mm}
\pagestyle{fancy}
\setlength{\parindent}{0pt} %Zeileneinrückung

\fancyhf{} %Setzt voreingestellte Kopf-und Fußzeilen-Eigenschaften zurück

\lhead{\nouppercase{\leftmark}}
\chead{}
\rhead{\thepage}

\lfoot{}
\cfoot{}
\rfoot{}

\title{\vspace{0cm}{\Huge Fortgeschrittenen-Praktikum I:\\ \vspace{1cm} Nuclear Magnetic Resonance}}
\author{Saskia Bondza\\Simon Stephan}
\date{performed on 15.09.2016 and 16.09.2016}

\pretitle{%
  \begin{center}
  \LARGE
  \includegraphics[width=6cm,]{figures/siegel}\\[\bigskipamount]
}
\posttitle{\end{center}}

%neue Commands----------------------------------------------------------------------------------------------------------
\newcommand{\nab}{\vec{\nabla}} %direkter Befehl mit Vektorpfeil
\newcommand{\gra}[2]{
	\begin{minipage}{\textwidth}
		\centering
		\includegraphics[width=0.7\textwidth]{figures/#1.png}
		\captionof{figure}{#2}
	\end{minipage}
	}


%Titel,Inhalt----------------------------------------------------------------------------------------------------------

\begin{document}
\pagenumbering{gobble} %verstecke Seitenzahl
\maketitle
\newpage

\thispagestyle{empty}
\tableofcontents
\newpage

%Schreiben----------------------------------------------------------------------------------------------------------
\pagenumbering{arabic} %verstecke Seitenzahl
\section{Einleitung}

%The aim of this experiment is to determine the gyromagnetic ratio of the proton in Hydrogen as well as in Glycol. Furthermore we want to measure the nuclear magnetic moment of $^{19}F$ and the proton resonance frequency of a Hydrogen sample. An additional measurement to confirm the homogeneity of the magnetic field involved in this experiment is also performed using a Hall effect sensor. 

%In this experiment we analyze nuclear magnetic resonance. Nuclear magnetic resonance is a method which uses the Zeeman effect to measure the gyromagnetic ratio of particles. We use it in this experiment to determine the gyromagnetic ratio of the proton in Hydrogen as well as in Glycol. Furthermore we want to measure the nuclear magnetic moment of $^{19}F$ and the proton resonance frequency of a Hydrogen sample. Additionally we analyze the magnetic field involved in this experiment with a Hall effect sensor to confirm its homogeneity.

In diesem Versuch untersuchen wir Kernspinresonanz. Dieser Effekt beruht auf dem Zeeman-Effekt und beschreibt das Umklappen von Kernspins in einem Magnetfeld unter Absorption oder Emission von Photonen. Wir benutzen die Kernspinresonanz um das gyromagnetische Verhältnis des Protons einmal in Wasserstoff und einmal in Glykol zu bestimmen. Anschließend messen wir das kernmagnetische Moment von $^{19}$F und die Protonenresonanzfrequenz von Wasserstoff. Außerdem untersuchen wir das in diesem Versuch benutzte magnetische Feld mit einer Hall-Sonde, um dessen Homogenität zu bestätigen.


\newpage
\section{Theoretische Grundlagen}
\subsection{Spin und Kernspin}


\subsection{Magnetisches Moment}

\subsection{Zeeman-Effekt und Kernspinresonanz}
A nucleus with a magnetic moment brought into a magnetic field in z-direction has the Energy $E=-\vec\mu\cdot\vec B=-g_I\mu_km_IB$. This means that the energy states of the nuclei are split up by their nuclear spin. The difference between two neighboring Zeeman states is $\Delta E=g_I\mu_kB$.

Transitions between these states can occur through spontaneous or induced emission or absorption of photons. The energy which is needed to induce such a transition or is getting free by this transition is exactly the energy difference between the two states. So the photons of a radiation field used to induce transitions need to have the frequency $\nu=\frac{\Delta E}{h}=\frac{g_i\mu_kB}{h}=\frac{\gamma B}{2\pi}$. 


\newpage
\section{Experimental Set-Up}


\newpage
\section{Experimental Procedure}



\newpage
\section{Evaluation}


\newpage
\section{Summary}


\newpage
\section{Attachement}

\subsection{Tables}

%\subsubsection{$\alpha$-Plateau Samarium}
%\lstinputlisting[language=MATLAB]{Rohdaten/alphaPlateau_Sm.txt}


%\newpage
%\subsection{Quellcode (MATLAB)}
%\lstinputlisting[language=MATLAB]{Rohdaten/alpha.m}

\newpage
\subsection{Lab Book}
%\begin{minipage}{\textwidth}
%\centering
%\includegraphics[width=0.9\textwidth]{figures/IMG_20151002_141014.jpg}
%\end{minipage}

\newpage
\listoffigures

%Literatur----------------------------------------------------------------------------------------------------------

%\cite{les}
\newpage
\thispagestyle{empty}
\begin{thebibliography}{9}

%\bibitem{staat}
%  Tobijas Kotyk,
%  \emph{Versuche zur Radioaktivität im Physikalischen Fortgeschrittenen Praktikum an der Albert-Ludwigs-Universität Freiburg},
%  Albert-Ludwigs-Universität, Freiburg,
%  2005
  

  
%\bibitem{molmasse}
%  \emph{http://www.convertunits.com/molarmass/<ELEMENTNAME AUF ENGLISCH>}, Stand 28.09.2015
  

\end{thebibliography}

\end{document}